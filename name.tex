%%%%%%%%%%%%%%%%%%%%%%%%%%%%%%%%%%%%%%%%%%%%%%%%%%%%%%%%%%%%
%%%%%%%%%%%%%%%%%%%%%%%%%%%%%%%%%%%%%%%%%%%%%%%%%%%%%%%%%%%%
%%%%%%%%%%%%%%%%%%%%%%%%%%%%%%%%%%%%%%%%%%%%%%%%%%%%%%%%%%%%
%%%%%%%%%%%%%%%%%%%%%%%%%%%%%%%%%%%%%%%%%%%%%%%%%%%%%%%%%%%%
%%%%%%%%%%%%%%%%%%%%%%%%%%%%%%%%%%%%%%%%%%%%%%%%%%%%%%%%%%%%
\documentclass[9pt]{article}
\usepackage{epsfig}
\usepackage{times}
\usepackage{fancyhdr}
\usepackage{pslatex}
\usepackage{amsmath}
\usepackage{mathrsfs}
\usepackage[dvipsnames]{xcolor}
\usepackage[hidelinks]{hyperref}%renewcommand{\topfraction}{1.0}
\renewcommand{\topfraction}{1.0}
\renewcommand{\bottomfraction}{1.0}
\renewcommand{\textfraction}{0.0}
\setlength {\textwidth}{6.6in}
\hoffset=-1.0in
\oddsidemargin=1.00in
\marginparsep=0.0in
\marginparwidth=0.0in                                                                               
\setlength {\textheight}{9.0in}
\voffset=-1.00in
\topmargin=1.0in
\headheight=0.0in
\headsep=0.00in
\footskip=0.50in                                         
\setcounter{page}{31}
\begin{document}
\def\pos{\medskip\quad}
\def\subpos{\smallskip \qquad}
\newfont{\nice}{cmr12 scaled 1250}
\newfont{\name}{cmr12 scaled 1080}
\newfont{\swell}{cmbx12 scaled 800}
%%%%%%%%%%%%%%%%%%%%%%%%%%%%%%%%%%%%%%%%%%%%%%%%%%%%%%%%%%%%
%     DO NOT CHANGE ANYTHING ABOVE THIS LINE
%%%%%%%%%%%%%%%%%%%%%%%%%%%%%%%%%%%%%%%%%%%%%%%%%%%%%%%%%%%%
%     DO NOT CHANGE ANYTHING ABOVE THIS LINE
%%%%%%%%%%%%%%%%%%%%%%%%%%%%%%%%%%%%%%%%%%%%%%%%%%%%%%%%%%%%
%     DO NOT CHANGE ANYTHING ABOVE THIS LINE
%%%%%%%%%%%%%%%%%%%%%%%%%%%%%%%%%%%%%%%%%%%%%%%%%%%%%%%%%%%%

\begin{center}
{\large
PHYS 20323/60323: Fall 2024-LaTeX Example
}\\
%%%%%%%%%%%%%%%%%%%%%%%%%%%%%%%%%%%%%%%%%%%%%%%%%%%%%%%%%%%%
{\large Citlali Alcala\ \star} \\[0.25in]
%%%%%%%%%%%%%%%%%%%%%%%%%%%%%%%%%%%%%%%%%%%%%%%%%%%%%%%%%%%%
\end{center}
%%%%%%%%%%%%%%%%%%%%%%%%%%%%%%%%%%%%%%%%%%%%%%%%%%%%%%%%%%%
% Section Heading
%%%%%%%%%%%%%%%%%%%%%%%%%%%%%%%%%%%%%%%%%%%%%%%%%%%%%%%

\textbf{1.}  
An electron is found to be in the spin state (in the z-basis):
\[\chi = A \begin{pmatrix} 3i \\ 4 \end{pmatrix}\]

%%%%%%%%%%%%%%%%%%%%%%%%%%%%%%%%%%%%%%%%%%%%%%%%%%%%%%%

\textbf{(a)(5 points)} Determine the possible values of \( A \) such that the state is normalized.

\vspace{1cm}


\textbf{(b)(5 points)} Find the expectation values of the operators \( S_x \), \( S_y \), \( S_z \), and \( \vec{\mathbf{S}}^2 \).

\vspace{1cm}

The matrix representations of the spin operators in the z-basis are given by:
\[
\textcolor{red}{S_x = \frac{\hbar}{2} \begin{pmatrix} 0 & 1 \\ 1 & 0 \end{pmatrix}, \quad }
\textcolor{purple}{S_y = \frac{\hbar}{2} \begin{pmatrix} 0 & -i \\ i & 0 \end{pmatrix}, \quad}
\textcolor{orange}{S_z = \frac{\hbar}{2} \begin{pmatrix} 1 & 0 \\ 0 & -1 \end{pmatrix}}
\]

\vspace{1cm}

%%%%%%%%%%%%%%%%%%%%%%%%%%%%%%%%%%%%%%%%%%%%%%%%%%%%%%%

\textbf{2.} The average electrostatic field in the Earth's atmosphere in fair weather is approximately given by:

%%%%%%%%%%%%%%%%%%%%%%%%%%%%%%%%%%%%%%%%%%%%%%%%%%%%%%%

\begin{center}
\[
\vec{\mathbf{E}} = E_0 \left( A e^{-\alpha z} + B e^{-\beta z} \right) \hat{z}
\]
\end{center}

\vspace{1cm}

\text{where \( A \), \( B \), \( \alpha \), \( \beta \) are positive constants, and \( z \) is the height above the (locally flat) Earth's surface.}  

\vspace{1cm}

\text{(a) (5 points)} Find the average charge density in the atmosphere as a function of height.

\vspace{1cm}

\text{(b) (5 points)} Find the electric potential as a function of height above the Earth.

%%%%%%%%%%%%%%%%%%%%%%%%%%%%%%%%%%%%%%%%%%%%%%%%%%%%%%%%%%%%%%%%%%%%%%%%%%%%%%%%%%%%%%%%%%%%%%%%%%%%%%%%%%%%%%

\vspace{1cm}
%%%%%%%%%%%%%%%%%%%%%%%%%%%%%%%%%%%%%%%%%%%%%%%%%%%%%%%
\textbf{3.}\textbf{The following questions refer to stars in the Table below.}


\textbf{}Note: There may be multiple answers.

\begin{table}[ht]
    \centering
    \hspace{2cm}
    \begin{tabular}{|c|c|c|c|c|c|}
    \hline
    Name & Mass & Luminosity & Lifetime & Temperature & Radius\\
    \hline
    $\beta$ Cyg. & $1.3\ M _\odot$ & $3.5 \ L_\odot$ & & & \\ \hline
    $\alpha$ Cen. & $1.0\ M _\odot$ & & & & $1\ R_\odot$ \\ \hline
    $\eta$ Car. & $60.\ M _\odot$ & $10^6 \ L_\odot$ & $8.0\times 10^5$ years & & \\ \hline
    $\epsilon$ Eri. & $6.0\ M _\odot$ & $10^3 \ L_\odot$ & & $20,000$ K & \\ \hline
    $\delta$ Scu. & $2.0\ M _\odot$ & &  $5.0 \times 10^8$ years & & $2\ R_\odot$\\ \hline
    $\gamma$ Del. & $0.7\ M _\odot$ & & $4.5 \times 10^{10}$ years & $5000$ K &\\ \hline
    \end{tabular}
    \label{tab:my_label}
\end{table}
%%%%%%%%%%%%%%%%%%%%%%%%%%%%%%%%%%%%%%%%%%%%%%%%%%%%%%%
\text{(a) (4 points)} Which of these stars will produce a planetary nebula.
%%%%%%%%%%%%%%%%%%%%%%%%%%%%%%%%%%%%%%%%%%%%%%%%%%%%%%%

\vspace{1cm}

%%%%%%%%%%%%%%%%%%%%%%%%%%%%%%%%%%%%%%%%%%%%%%%%%%%%%%%
\text{(b) (4 points)} Elements heavier than Carbon will be produced in which stars.
%%%%%%%%%%%%%%%%%%%%%%%%%%%%%%%%%%%%%%%%%%%%%%%%%%%%%%%

\end{document}
